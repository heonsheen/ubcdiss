\chapter{Conclusion}
\label{ch:Conclusion}

Although our method generates more realistic volume preservation and
reduces locking, for simulations without large deformations, the
standard Neo-Hookean model may be sufficient due to its
simplicity. This is especially true when no other constraints, such as due to external contact, are present in the simulation. Constrained optimization adds
some complexity to the simulation, though our results in Table
\ref{tab:performance} show that the increase is computation times is
not prohibitive in most cases. Overall, our findings allow a simple and inexpensive
extension to existing FEM systems to solve the problem of volumetric
locking while simulating incompressible materials such as the human
body.

\paragraph{Performance} Our formulation uses exact non-linear volume constraints on a non-linear
optimization problem to preserve volume exactly in demanding applications like statics and dynamics
with large time steps.  This limits the choice of optimization solvers to ones that
support non-linear equality constraints (e.g. Interior Point or SQP solvers).  However, for dynamics problems
with smaller time steps, or in applications with tolerance for volume loss/gain, we recommend
linearizing the volume constraint, which drastically simplifies the problem. This can reduce the
overhead of enforcing equality constraints, while still avoiding locking.

\paragraph{Choosing $\lambda$} By decoupling $\lambda$ from its physical meaning, our formulation is
faced with an additional challenge, which is to determine how exactly $\lambda$ effects the outcome
of a simulation.  Fortunately, this is not a significant drawback since material parameters for
standard Neo-Hookean FEM simulations also deviate from their measured values due to numerical
stiffenning. This means that even the parameters of standard models require manual tuning to
reproduce real phenomena in simulation.  Luckily data-driven methods for determining simulation
parameters (which has seen significant attention in recent literature) are generally agnostic to the
true physical meaning of these parameters, and thus are equally as compatible with our method.

In conclusion, we presented a general method for realistic volumetric FEM simulations of human tissue.
Our method provides exact volume preservation without the artificial stiffness due to volumetric
locking using zonal volume constraints. This method gives artists the ability to define volume
preserving zones that conform to anatomical compartments and automatically produces
``squash-and-stretch'' effects. In addition, we introduced an epidermis model for simulating
skin dynamics, as an additional surface area-preserving potential. We also proposed a
modification to the energy potential to provide control over local volume flow
that results in improved
recovery during extreme compression and inversion. Our approach can be applied to a variety of energy
models. In particular, we have demonstrated the effectiveness of these simple modifications to the
invariant based non-linear hyperelastic energies like the Neo-Hookean and Stable Neo-Hookean energy
models.
