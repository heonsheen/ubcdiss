\chapter{Related Work}
\label{ch:Related Work}

A number of recent contributions have significantly improved the
performance and behavior of hyperelastic solid simulations.  The
standard approach is using the Finite Element Method (FEM) on a
Lagrangian tetrahedral (or hexahedral) mesh \cite{Sifakis:2012}.  The
methods used for soft tissue simulations are typically split between
linear and non-linear hyperelasticity models.  A number of popular
elasticity models are used for soft tissue simulation including
Neo-Hookean, St. Venant-Kirchhoff, and co-rotated elasticity.

Co-rotated elasticity \cite{Muller:2002,McAdams:2011}, has been tremendously successful in real-time
and interactive applications largely due to its simplicity.  However, it suffers from element
degeneration in large deformations \cite{Civit:2014} and has poor volume preservation properties
\cite{Smith:2018}.  Non-linear energy models like the Neo-Hookean models \cite{BonetWood:2008}, have
been used to circumvent these issues at a larger computational cost; although in recent years,
Neo-Hookean elasticity has also appeared in interactive simulations \cite{Liu:2017}.

Our work targets invariant based non-linear hyperelastic models, such as Neo-Hookean elasticity, for
their generality, superb handling of large deformations and inherent reflection stability.

Among non-linear elastic models are compressible and incompressible hyperelastic models. While
incompressible models \cite{Mooney:1940,Rivlin:1948} pose an explicit volume constraint on each
element, compressible models prevent severe compression using a penalty term \cite{BonetWood:2008}.
The most popular method for solving elasticity problems in computer graphics is the standard linear
FEM on a Lagrangian mesh because of its performance profile and versatility.
Unfortunately, imposing a severe penalty --- let alone a hard constraint --- for volume change on
each element can cause severe numerical instabilities and locking especially in linear tetrahedral FEM.
% \todo{Need a reference to support this claim}.

In traditional FEM, volumetric locking is addressed by decoupling incompressibility from displacements with Mixed Finite Element Methods. 
In computer graphics, Irving et al. \cite{Irving:2007} have addressed locking in tetrahedral meshes by
softly constraining the volume of the one ring around each vertex in a tetrahedral mesh using position
and velocity correction steps. This approach is an application of nodal strain elements \cite{bonet:1998},
where stress and stain, in this case their volumetric component, is nodally interpolated in a mixed FEM. 
However, without additional stabilization of some sort, these types of mixed elements are known to be unstable 
and the number of additional pressure variables are proportional to the number of nodes.
A further discussion of Mixed Finite Element Methods was presented in Chapter~\ref{ch:Background}.
Kaufmann et al. \cite{Kaufmann:2012} looked to solve locking by introducing additional degrees of freedom 
to the system by using a Discontinuous Galerkin discretization. Smith et al. \cite{Smith:2018} proposed 
the Stable Neo-Hookean energy model to handle invertible elements as well as improve stability for high Poisson's ratios.

Irving et al. \cite{Irving:2007} was an implementation of the 
Average Nodal Pressure element \cite{bonet:1998}, where the cell-wise constant pressure samples of the
one-ring neighbors are averaged on the nodes. By contrast, our method uses pressure samples
(i.e. the volume preserving zones) that are coarser and decoupled from the mesh topology.
Fine-scale cell-wise pressures are instead controlled through a local penalty, which avoids
additional pressure variables.  This way we can keep $\lambda$ low enough to
avoid locking, while simultaneously enforce volume preservation.
Our method has significantly fewer constraints, compared to the total number of vertices, which
permits enforcing constraints exactly using constraint minimization to
solve the variational problem rather than using constraint projection.

Some works have targeted global volume preservation
\cite{Hong:2006,Hirota:2000,Promayon:1996,Diziol:2011}, however not in the context of volumetric
FEM.  Global volume constraints have also been applied in studies in skinning methods
\cite{Rohmer:2009}. Some also proposed using a sweep-based approach to conserve the volume
\cite{Yoon:2006,Angelidi:2004} or a vector field approach \cite{Funck:2007}. In contrast, we propose
zonal volume constraints for Neo-Hookean type energy models for Lagrangian FEM simulations.

Finite element simulations also suffer from element inversions during severe deformation.
Inversion stability allows simulations to handle large deformations and permits taking large time
steps, which can improve simulation performance significantly.
A line of recent work has proposed methods for resolving element inversions by extending the energy
density function to the negative volume region.  Force filtering methods
\cite{Irving:2004,Teran:2005} have improved inversion stability but suffer from subtle problems
including invalid inversion recovery directions or derivative drift as thoroughly explored in
\cite{Smith:2018}. Stomakhin et al. \cite{Stomakhin:2012} propose a $C^1$ or $C^2$ extension of the
entire energy density function for low volume fractions, which resolves many of these problems.
However, filtering methods can be quite sensitive to appropriate specification of filtering thresholds
and reflection conventions \cite{wang:2016}. We instead follow a simpler approach similar to \cite{Smith:2018},
where we design a volumetric penalty term to satisfy necessary conditions for stability and inversion
robustness. Our penalty function improves upon the Stable Neo-Hookean volumetric term by introducing nonlinearity
to the stress also, resulting in better inversion recovery and improved performance.