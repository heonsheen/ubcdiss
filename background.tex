\chapter{Background}
\label{ch:Background}

\section{Variational Elasticity}
In this section, we establish the context for our contributions by introducing FEM simulation of
hyperelastic materials as a variational problem.

Let $\Omega \subset \R^{3}$ be a union of mesh tetrahedra representing an elastic solid in its
undeformed configuration. Then let $\vec{x} \in \R^{3n}$ correspond to a stacked vector of mesh vertex
positions that prescribe the deformation of the solid, where $n$ is the total number of vertices
in the mesh. In an elasticity problem, we are
interested in finding the configuration $\vec{x}$ that results in the lowest potential energy
for the elastic solid $\Omega$ given a set of boundary conditions
and external forces. Mathematically, we may write the problem statement as
\begin{align}
\vec{x}^* := \argmin_{\vec{x}} W(\vec{x}),
\label{eq:energy_min}
\end{align}
where $W(\vec{x})$ represents the elastic work function for configuration $\vec{x}$. 
This formulation allows conservative external forces to be added as additional potentials in the
objective, however for the sake of simplicity we ignore external forces in the following sections.

With linear (constant strain) elements, $W$, which is the integral of energy density function $\Psi(\vec{x})$, can be written as the sum of volume scaled per-element energies:
\begin{align}
W(\vec{x}) = \int_{\Omega} \Psi(\vec{x}) := \sum_{e} V_{e} \Psi(\vec{F}_e(\vec{x})),
\label{eq:total_energy}
\end{align}
where $V_e$ is the volume of element $e$ in the reference configuration, which depends on the element deformation gradient $\vec{F}_e$.  The choice of the energy density function $\Psi$ determines the hyperelastic energy model. 

For the time discretization, we adopt the implicit Euler time integrator and add an inertial
energy term to this minimization, however in this thesis we focus primarily on static FEM for
simplicity. 

\section{Incompressibility and Locking}
There are two ways in which incompressibility could be enforced: either directly, as a constraint that the volume is preserved, or indirectly with a penalty term that powerfully resists compression. Since both ways are frequently referred to as ``incompressible,'' to avoid confusion we will refer to incompressible Neo-Hookean models using the first method as {\em ``Constrained Neo-Hookean''} (CNH), and those using the second method as
{\em ``Unconstrained Neo-Hookean''} (UNH).

Most incompressible hyperelastic energy models used in graphics are of
the Unconstrained Neo-Hookean type, and penalize element-wise volume change with a term scaled by
the first Lam\'e parameter $\lambda$, which depends on the Young's Modulus $E$ and Poisson's Ratio
$\nu$. For instance, the most common version of such an energy density function
\cite{BonetWood:2008} is written as
\begin{align}
\Psi_{\text{UNH}}(\vec{F}; \lambda, \mu) = \frac{\mu}{2}(I_C - 3) - \mu \log J + \frac{\lambda}{2} (\log J)^2
\label{eq:neohookean}
\end{align}
where $I_C = \tr(\vec{F}^{\top}\vec{F})$ and $J = \det(\vec{F})$, which represents the fraction of
volume after deformation. This means that when $J$ is close to zero (extreme compression),
$\Psi_{\text{UNH}}$ will generate large penalty forces to restore the element to reference
configuration. Another commonly used material model, co-rotated elasticity \cite{McAdams:2011} is written as
\begin{align*}
\Psi_{\text{CR}}(\vec{F}; \lambda, \mu) = \mu\|\vec{F} - \vec{R}\|^2_F + \frac{\lambda}{2} \tr(\vec{S} - \vec{I})^2,
\end{align*}
where $\vec{R}$ and $\vec{S}$ form the polar decomposition: $\vec{F} = \vec{R}\vec{S}$ and $\vec{I}$
is the $3\times3$ identity matrix. Here, in a similar fashion, local compression is once again
penalized by $\lambda$. However, as discussed in the introduction, high Poisson's ratios make the
system stiff, which results in stiffness related issues such as instability, and artificial damping.

\section{Mixed Finite Element Methods}

It is often necessary to compute reliable solutions not only for displacements but also for
pressures (e.g., for frictional contact or fractures). For displacement-based one-field FEM,
pressure must be computed from the displacement variables $\vec{x}$. Specifically, cell-wise
hydrostatic pressure is usually computed as the negative of the divergence of the Cauchy stress
tensor. Since the Cauchy stress tensor is related to the derivative of the energy density function
$\Psi(\vec{x})$, essentially the pressures computed from a one-field FEM mainly depend on the volume
term of $\Psi$. However, due to similar issues as discussed above, when the material is
incompressible and $\lambda \rightarrow +\infty$ the volume term stops being a reliable model for
volumetric stress.

One traditional way of decoupling incompressibility from $\lambda$ is by introducing an
additional pressure variable $\vec{p}$ that models the volumetric stress component of elements,
interpolated separately from displacement on the finite element mesh \cite{Bathe:2006}. This allows us to reformulate the variational problem as
\begin{align}
\vec{x}^* &:= \arg \max_{\vec{p}} \min_{\vec{x}} \int_{\Omega} \hat{\Psi}(\vec{x}) + \vec{p}^T \vec{c}(\vec{x}),
\label{eq:mixed_variational}
\end{align}
where $\hat{\Psi}(\vec{x})$ is the deviatoric component of the displacement-based elastic potential,
and $\vec{c}(\vec{x})$ is a term that relates $\vec{p}$ to $\vec{x}$. This additional term can be
interpreted as a constraint on $\vec{p}$ to be proportional to the hydrostatic pressure computed
from the displacements $\vec{x}$. Then, $\vec{p}$ becomes the Lagrange multiplier for the constraint
$\vec{c}(\vec{x})$.  One implementation of this type of formulation is shown in 
\cite{Sussman:1987}.  These methods are known as the displacement-pressure Mixed Finite Element
Methods and are one of the most accurate ways to solve the problem.

With the additional degree of freedom, the Bab\u{u}ska-Brezzi inf-sup condition restricts the 
choice of the space of finite element basis for the additional variable for the method to be stable \cite{bathe:2001}. 
This condition dictates that the order of basis for the displacement variables must be higher than that
of the pressure 
variables. Specifically, for conforming tetrahedral elements the lowest order finite element space choices are either 
the Hood-Taylor elements ($P_2$ for $\vec{x}$ / $P_1$ for $\vec{p}$, where $P_k$ denotes the space of $k$-th order polynomials), or MINI ($P_1^+ / P_1$, where superscript $+$ denotes an enrichment of cubic bubble \cite{arnold:1984}).
Hence, Mixed FEM with a simple linear tetrahedral Finite Element basis for displacement is usually not valid for stable simulations. 
This includes the Average Nodal Pressure elements proposed in \cite{Irving:2007}, where the Lagrange multipliers of 1-ring volume constraints can be interpreted as cell-wise constant pressure variables ($P_0$) being averaged on the nodes. Although this alleviates some of the problems arising from each element being constrained, it still fails to meet the inf-sup condition and spurious modes may occur without additional stabilization \cite{puso:2006}. 

Recent research has been focused on developing a stabilized low-order tetrahedral element for
incompressible elasticity \cite{scovazzi:2016}. Many of these methods allow almost a $P_1 / P_1$
mixed element to be stable and accurate, by adding an additional stabilization term to the FEM basis. However, these methods are still much more expensive than standard linear tetrahedral elements, since the additional pressure variables cannot be solved with simple constraints and require a specific mixed FEM system to be built and solved.
The intuition for our approach from these works is that, to achieve an efficient and stable computation of the additional pressure variables, one must sample the pressure variables in a coarser scale compared to the displacement variables. Then, we are able to split the pressure computation into a coarser and finer scale, to control the coarse-scale pressures as separate pressure variables as Lagrange multipliers for volume constraints, as in Eq.\ref{eq:mixed_variational}, and compute the fine-scale pressures from displacements. Therefore, we look to a much more efficient and simpler approach by enforcing a volume constraint for a few larger zones of elements, and modeling the element-wise local pressure as an additional penalty term.
